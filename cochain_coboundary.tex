\documentclass[titlepage,12pt]{article}
\usepackage{amsmath,amsthm,amssymb,ascmac,amscd}
\usepackage[dvipdfmx]{graphicx}
\newtheorem{defi}{Definition}[section]
\newtheorem{ex}{Example}[section]
\newtheorem{lem}{Lemma}[section]
\newtheorem{thm}{Theorem}[section]
\newtheorem{rem}{Remark}[section]
\newtheorem{prop}{Proposition}[section]
\newcommand{\mm}{\mathfrak{m}}
\newcommand{\g}{\mathfrak{g}}
\newcommand{\m}{\mathcal{V}_{n}}
\newcommand{\w}{\wedge}
\newcommand{\la}{\langle}
\newcommand{\ra}{\rangle}
\renewcommand{\proofname}{\textbf{Proof}}
\begin{document}

\begin{defi}
Let $\mathfrak{g}$ be a Lie algebra on a field $K$ and $V$ be an n-dimensional vector space over $K$. And, Let $\phi :
\mathfrak{g} \times V \longrightarrow V$ be a $K$-bilinear map. Hereinafter, we write
$\phi(x,v)=x \cdot v(\forall x \in \g, \forall v \in V)$. Then, ($V,\phi$) is $\g$-module such that
\begin{equation*}
\phi([x,y],v) = [x,y] \cdot v = x \cdot (y \cdot v)-y \cdot (x
\cdot v) \quad \forall x,y \in \g, \forall v \in V.
\end{equation*}
\end{defi}

\begin{defi}
Let $\mathfrak{g}$ be a Lie algebra and $V$ be a $\mathfrak{g}$-module. Then, 
a $p$-dimensional \textbf{cochain} of $\mathfrak{g}$ with values in $V$ is a $p$-linear alternating mapping from 
$\underbrace{\mathfrak{g} \times \mathfrak{g} \times \cdots
\times \mathfrak{g}}_{p}$ to $V$. \\
\ However, if $p$ is $0$, $0$-cochain of $\mathfrak{g}$ is a constant mapping from $\mathfrak{g}$ to $V$.
\end{defi}

\ Hereinafter, let $\g,V$ be finite dimension and $C^{p}(\mathfrak{g},V)$ be a $p$-cochain space of $\mathfrak{g}$. That is,
\begin{alignat*}{2}
C^{p}(\g,V) &\cong \operatorname{Hom}(\Lambda^{p}\g,V), \quad
\phi \mapsto e_{i_{1}} \wedge \dots \wedge e_{i_{p}} \mapsto
\phi(e_{i_{1}},\dots,e_{i_{p}}) & \qquad & (p \geqq 1) \\
C^{0}(\g,V) &= V & \qquad & (p = 0) \\
C^{p}(\g,V) &= {0} & \qquad & (p<0)
\end{alignat*}
($e_{1},\dots,e_{n}$ are basis of $\g$, $1 \leqq i_{1} < \dots < i_{p} \leqq n$). Also,
let $C^{*}(\g,V)$ be a direct sum of all $C^{p}(\g,V)$. Then, we will define $\theta(x)$,
$i(y)$ as follows. Firstly, we will confirm that for $x,x_{1},\dots,x_{p}$ in $\g$ and $\Phi$ in $C^{p}(\g,V)$,
\begin{equation*}
(x \cdot \Phi)(x_{1},\dots,x_{p})=x \cdot
\Phi(x_{1},\dots,x_{p}) - \sum_{1\leqq i \leqq p} \Phi
(x_{1},\dots,x_{i-1},[x,x_{i}],\dots,x_{p}) \tag{$\ast$}
\end{equation*}
defines a $\g$-module structure in $C^{p}(\g,V)$. In fact,
\begin{align*}
&(x \cdot(y \cdot \Phi))(x_{1},\dots,x_{p})-(y \cdot (x \cdot
\Phi))(x_{1},\dots,x_{p}) \\
&= x \cdot ((y \cdot \Phi)(x_{1},\dots,x_{p})) - \sum_{i}(y
\cdot \Phi)(x_{1},\dots,[x,x_{i}],\dots,x_{p}) \\
& -y \cdot((x \cdot \Phi)(x_{1},\dots,x_{p}))+\sum_{i}(x \cdot
\Phi)(x_{1},\dots,[y,x_{i}],\dots,x_{p}) \\
&=x \cdot \{y \cdot
(\Phi(x_{1},\dots,x_{p}))-\sum_{j}\Phi(x_{1},\dots,[y,x_{j}],\dots,x_{p})\}
\\
& -\sum_{i}\{y \cdot (\Phi(x_{1},\cdots,[x,x_{i}],\dots,x_{p}) -
\sum_{j \neq
i}\Phi(x_{1},\dots,[y,x_{j}],\dots,[x,x_{i}],\dots,x_{p}) \\
& -\Phi(x_{1},\dots,[y,[x,x_{i}]],\dots,x_{p})\} \\
& -y \cdot \{x \cdot (\Phi(x_{1},\dots,x_{p})) -
\sum_{j}\Phi(x_{1},\dots,[x,y_{j}],\dots,x_{p}\} \\
& -\sum_{i}\{x \cdot
\{\Phi(x,\cdots,[y,x_{i}],\dots,x_{p})-\sum_{i \neq
j}\Phi(x_{1},\dots,[x,x_{j}],\dots,[y,x_{j}],\dots,x_{p}) \\
& -\Phi(x_{1},\dots,[x,[y,x_{j}]],\dots,x_{p})\} \\
&= [x,y] \cdot
(\Phi(x_{1},\dots,x_{p}))-\sum_{i}\Phi(x_{1},\dots,[x,y],x_{i}],\dots,x_{p})
\\
&=([x,y] \cdot \Phi)(x_{1},\dots,x_{p}).
\end{align*}

We write $\theta$ as a representation corresponding to this $(\ast)$. Also, for $y$ in $\g$,
we define $i(y)$ as a homomorphism from $C^{p}$ to $C^{p-1}$ such as the following.
\begin{equation*}
(i(y)\Phi)(x_{1},\dots,x_{p-1}) = \Phi(y,x_{1},\dots,x_{p-1}).
\end{equation*}
\begin{lem}
For $x,y$ in $\g$,
\begin{equation*}
\theta(x) \circ i(y) - i(y) \circ \theta(x) = i([x,y]) .
\end{equation*}
\end{lem}

\begin{proof}
Substituting $\Phi$,
\begin{align*}
(Left side) &= ((\theta(x) \circ i(y))\Phi - (i(y) \circ
\theta(x))\Phi)(x_{1},\dots,x_{p-1}) \\
&= (\theta(x)(i(y)\Phi))(x_{1},\dots,x_{p-1}) -
(i(y)(\theta(x)\Phi))(x_{1},\dots,x_{p-1}) \\
		&= x(i(y)\Phi)(x_{1},\dots,x_{p-1}) \\ 
& - \sum_{1 \leqq i \leqq
p-1}(i(y)\Phi)(x_{1},\dots,x_{i-1},[x,x_{i}],\dots,x_{p-1}) -
(\theta(x)\Phi)(y,x_{1},\dots,x_{p-1}) \\
&= x\Phi(y,x_{1},\dots,x_{p-1}) - \sum_{1 \leqq i \leqq
p-1}\Phi(y,x_{1},\dots,x_{i-1},[x,x_{i}],\dots,x_{p-1}) \\
& - \bigl\{ x\Phi(y,x_{1},\dots,x_{p-1}) -
\Phi([x,y],x_{1},\dots,x_{p-1}) \\ & - \sum_{1 \leqq i \leqq
p-1}\Phi(y,x_{1},\dots,x_{i-1},[x,x_{i}],\dots,x_{p})\bigr\} \\
		&=\Phi([x,y],x_{1},\dots,x_{p-1}) \\
		&=(i([x,y])\Phi)(x_{1},\dots,x_{p-1}) = (Right side).
\end{align*}
\end{proof}

\begin{defi}
Let $\g$ be a Lie algebra and $V$ be $\g$-module. Then, we define endomorphism $d$ as 
\begin{align*}
d &: C^{*}(\g,V) \longrightarrow C^{*}(\g,V) \\
d\Phi(x) &= x \cdot \Phi \quad \operatorname{for} \quad \forall \Phi
\in C^{0}(\g,V), \ \forall x \in \g \\
d\Phi(x_{1},\dots,x_{p+1}) &= \sum_{1 \leqq s \leqq
p+1}(-1)^{s+1} x_{s} \cdot
(\Phi(x_{1},\dots,\hat{x}_{s},\dots,x_{p+1})) \\
& + \sum_{1 \leqq s < t \leqq
p+1}(-1)^{s+t}\Phi([x_{s},x_{t}],x_{1},\dots,\hat{x}_{s},\dots,\hat{x}_{t},\dots,x_{p+1})
\\
& \operatorname{for} \quad \forall \Phi \in C^{p}(\g,V)(p \geqq 1), \,
\forall x_{1},\dots,x_{p+1} \in \g.
\end{align*}
$d$ maps an element of $C^{p}(\mathfrak{g},V)$ to $C^{p+1}(\mathfrak{g},V)$. This $d$ is called \textbf{coboundary operator}.
\end{defi}

\begin{thm}
$d$ satisfies $d^{2}=0$.
\end{thm}
Fistly, we will prove this lemma.
\begin{lem} 
\begin{align*}
&\operatorname{(i)} \quad \theta(x) = i(x) \circ d + d \circ
i(x) \\
&\operatorname{(ii)} \quad d \circ \theta(x) = \theta(x) \circ d
\quad \operatorname{for} \quad \forall x \in \g.
\end{align*}
\end{lem}

\begin{proof}
About $\operatorname{(i)}$, \\
\ For $x,x_{1},\dots,x_{p}$ in $\g$,
\begin{align*}
((i(x) \circ d)(\Phi))(x_{1},\dots,x_{p}) &=
(i(x)(d\Phi))(x_{1},\dots,x_{p}) \\ &=
d\Phi(x,x_{1},\dots,x_{p}) \\
&= x \cdot (\Phi(x_{1},\dots,x_{p})) - \sum_{1 \leqq s \leqq
p}(-1)^{s+1}\theta(x_{s})\Phi((x,x_{1},\dots,\hat{x}_{s},\dots,x_{p}))\\
& + \sum_{1 \leqq s \leqq
p}(-1)^{s+1}\Phi([x,x_{s}],x_{1},\dots,\hat{x}_{s},\dots,x_{p})
\\
& + \sum_{1 \leqq s<t \leqq
p}(-1)^{s+t}\Phi([x_{s},x_{t}],x,x_{1},\dots,\hat{x}_{s},\dots,\hat{x}_{t},\dots,x_{p})
\\
	&= x \cdot (\Phi(x_{1},\dots,x_{p})) \\
& - \Bigl\{ \sum_{1 \leqq s \leqq
p}(-1)^{s+1}\Bigl(i(x)(\theta(x_{s})\Phi)-i([x,x_{s}])\Phi\Bigr)(x_{1},\dots,\hat{x}_{s},\dots,x_{p})
\\
& +\sum_{1 \leqq s<t \leqq p
}(-1)^{s+t}(i(x)\Phi)([x_{s},x_{t}],x_{1},\dots,\hat{x}_{s},\dots,x_{p})
\Bigr\} \\
&= (\theta(x)-d \circ i(x))(\Phi(x_{1},\dots,x_{p})).
\end{align*}
About $\operatorname{(ii)}$, let $\Phi$ be a 0-cohchain. Then,
\begin{align*}
(\theta(x) \circ d\Phi)(y)
	&= x(y\Phi)-[x,y]\Phi \\
	&= x(y\Phi)-{x(y\Phi)-y(x\Phi)} 
	= y(x\Phi) = (d\theta(x)\Phi)(y) \\
	&\therefore \  \theta(x)d\Phi=d\theta(x)\Phi
\end{align*}
Now, we assume that this relationship holds up to $p-1$. That is, for $\Phi$ in $C^{p-1}(\g,V)$,
$d(\theta(x)\Phi) = \theta(x)d\Phi$.
Assuming that $\Phi$ is an element of $C^{p}(\g,V)$,
\begin{align*}
(i(y)(d\theta(x)-\theta(x)d))\Phi &= (i(y)(d\theta(x))\Phi -
i(y)\theta(x)d\Phi \\
&= (\theta(y)-di(y))\theta(x)\Phi +
(i([x,y])-\theta(x)i(y))d\Phi \\
&= \theta(y)\theta(x)\Phi - di(y)\theta(x)\Phi + i([x,y])d\Phi -
\theta(x)i(y)d\Phi \\
&= \theta(y)\theta(x)\Phi + d(i([x,y])-\theta(x)i(y))\Phi \\
& + \bigl\{\theta([x,y])-di([x,y])\bigr\}\Phi -
\bigl\{\theta(x)(\theta(y)-di(y))\bigr\}\Phi \\
&= (\theta(x)d-d\theta(x))(i(y)\Phi) = 0 \\
&\therefore \ (d\theta(x)-\theta(x)d)\Phi = 0.
\end{align*}
\end{proof}
then, we will prove theorem 0.1.
\begin{proof}
We prove by induction on $p$. Firstly, for $\Phi \in C^{0}(\g,V), \forall x,y \in \g$,
\begin{align*}
((d \circ d)(\Phi))(x,y) &= x \cdot (d\Phi(y))-y \cdot
(d\Phi(x)) - d\Phi([x,y]) \\
&=0.
\end{align*}
Secondary, we suppose that the above equation holds up to $p-1$. Then, for $\Phi \in C^{p}(\g,V),
x_{1},\dots,x_{p+2} \in \g$,
\begin{align*}
((d \circ d)(\Phi))(x_{1},\dots,x_{p+2}) &=(i(x_{1}) \circ d
\circ d)(\Phi)(x_{2},\dots,x_{p+2}) \\
&= ((\theta(x_{1})-d \circ i(x_{1})) \circ
d)(\Phi)(x_{2},\dots,x_{p+2}) \\
&=(\theta(x_{1}) \circ d - d \circ (\theta(x_{1}) - d \circ i
(x_{1})))(\Phi)(x_{2},\dots,x_{p+2}) \\
&=((d \circ d \circ i(x_{1})(\Phi)(x_{2},\dots,x_{p+2}) \\
&= (d \circ d)(i(x_{1})(\Phi))(x_{2},\dots,x_{p+2})
\end{align*}
Therefore, $d^{2}=0$.
\end{proof}


\begin{thebibliography}{99}
\item
L. Boza, E. M. Fedriani, J. N\'{u}\~{n}ez, and \'{A}. F. Tenorio. \textit{A Historical Review of the Classifications of Lie Algebra}. Rev. Un. Mat. Argentina, \textbf{54}, No 2, 2013.
\item
K. Erdmann and Mark J .Wildon. \textit{Introduction to Lie Algebras}. Springer, 2006.
\item
D. B. Fucks. \textit{Cohomology of Infinite Dimensional Lie Algebras}. Plenum Publishing Corporation, 1986.
\item
M. Goze and Y. Khakimdjanov. \textit{Nilpotent Lie Algebra}. Springer Science+Business Media, B.V., Volume 361, 1996.
\item
P. Griffiths and J. Harris. \textit{Principles of Algebraic Geometry}. John Wiley & Sons, 1978.
\item
J. McCleary. \textit{A User's Guide to Spectral Sequences}, 2ED. Cambridge University Press, 2001.
\item
D. V. Millionshchikov. \textit{Deformations of Filiform Lie Algebras and Symplectic Structures}. Proceedings of the Steklov Institute of Mathematics, \textbf{252}, 182-204, 2006.
\item
M. Vergne. \textit{Cohomologie des alg\`{e}bres de Lie nilpotents}. Bull. Soc. Math. France \textbf{98}, 81-116, 1970.
\end{thebibliography}


\end{document}